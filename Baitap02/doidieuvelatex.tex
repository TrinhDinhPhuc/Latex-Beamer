\chapter{Đôi điều về \LaTeX}
\section{\LaTeX là gì?}
\begin{itemize}
	\item \LaTeX ~là chương trình soạn thảo văn bản hay bài trình chiếu giống như Word, Powerpoint.
	\item Tiền thân là TeX, một ngôn ngữ định dạng do Donald Knuth phát minh, rất khó sử dụng.
	\item LaTeX được phát minh bởi Leslie Lamport, dựa trên TeX nhưng dễ sử dụng hơn rất nhiều. Cho ra chất lượng bản in cực cao cùng cấu trúc văn bản rất logic và đồng bộ.
	\item Tự động hóa rất nhiều so với làm bằng tay của Word.
\end{itemize}
\section{Ưu điểm của \LaTeX}
Đối với những người dùng máy tính, khi chuyển sang một môi trường soạn thảo mới, họ thường tìm hiểu xem môi trường soạn thảo mới có những tính năng gì đặc biệt. Tôi cũng vậy, khi chuyển sang \LaTeX, tôi thấy nó có nhiều ưu điểm hơn so với Word:
\begin{itemize}
	\item\textbf{Văn bản nhất quán:} khoảng cách dòng, kích cỡ chữ, màu sắc, cách trình bày,… cho dù qua văn bản khác nó cũng như vậy.
	\item \textbf{Hoàn toàn tự động:} đánh số chương, tiêu đề mẹ, tiêu đề con, đánh số phương trình, bảng, hình ảnh, tham chiếu$,\ldots$ hoàn toàn tự động.
	\item \textbf{Trích dẫn tài liệu tham khảo:} tự động và nhất quán, style rất đẹp.
	\item \textbf{Làm việc với một dự án lớn:} cả trăm, cả ngàn trang trong một file .tex dung lượng rất nhỏ, dễ quản lý và điều khiển.
	\item \textbf{Tích hợp công thức toán học:} công thức toán học tích hợp rất hài hòa với văn bản, đẹp và rõ nét.
	Vẽ hình đẹp: hình vẽ và chữ chú thích trên hình vẽ rất hài hòa với văn bản (cỡ chữ, ko bị vỡ nét khi zoom$,\ldots$)
\end{itemize}
\section{Nhược điểm của \LaTeX}
Song hành với những ưu điểm kể trên, \LaTeX ~còn có một số nhược điểm như sau:
\begin{itemize}
	\item Biên soạn các tài liệu không theo cấu trúc chung rất khó.
	\item Làm việc chỉ nhờ vào dòng lệnh, giống như lập trình nên thường xảy ra lỗi nhỏ nhặt nhưng tốn thời gian sửa.
	\item Cách học và tiếp cận tốn nhiều thời gian hơn.
\end{itemize}
