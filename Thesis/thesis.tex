%%14:53:45 30/6/2018 -VieTeX creates C:\Users\PhucCoi\Documents\Beamer\Thesis\thesis.tex
%%%%%%%%%%%
% Mẫu để các bạn làm slides trình chiếu khi thuyết trình.
% Tôi chỉ lấy phần đơn giản nhất của gói lệnh beamer hướng dẫn
% Các bạn tự tìm hiểu sau này.
% Tác giả tài liệu này: Nguyễn Hữu Điển
%  Khoa Toán - Cơ - Tin học
 % Đại học Khoa học Tự nhiên Hà nội
%Email: huudien@vnu.edu.vn, 
%%%%%%%%%%%%%
\documentclass[10pt]{beamer}
\usepackage{amsmath,amsxtra,amssymb,latexsym, amscd,amsthm}
\usepackage{indentfirst}
\usepackage[mathscr]{eucal}
\usepackage{color}
\usepackage[utf8]{vietnam}
%\usetheme{split}
\usepackage{beamerthemesplit}
%\usepackage{beamerthemesidebar}
%\usetheme{Warsaw}
% \usetheme{bergen}
%  \usetheme{Madrid}
 %\usetheme{boadilla}
%\setbeamercovered{transparent}

\usepackage[english]{babel}

\usepackage{listings}
\title{Dùng Beamer tạo trang thuyết trình}
\author{Nguyễn Hữu Điển}
%\subject{Presentation Programs}
\institute[ĐHKHTN Hà Nội]{
  Khoa Toán - Cơ - Tin học\\
  Đại học Khoa học Tự nhiên Hà nội}
\date{7/5/2006}
\begin{document}

\frame{\maketitle}

\begin{frame}
  \frametitle{1. Lấy và cài đặt gói lệnh Beamer}
  \begin{itemize}
  \item\textcolor[rgb]{1.0,0.0,0.50}{ Lấy gói lệnh từ trang web:\\
  http://latex-beamer.sourceforge.net}
    \pause
  \item\textcolor[rgb]{1.0,0.0,0.50}{ Mở gói lệnh latex-beamer-3.06.orig.tar vào thư mục
   C:$\backslash$texmf$\backslash$tex$\backslash$latex$\backslash$latex-beamer-3.06}
  \pause
  \item Gắn vào môi trường MikTeX bằng cách chạy chương trình:
 Start$\rightarrow$Programs $\rightarrow$ MikTeX $\rightarrow$ MikTeX options $\rightarrow$ refresh
\pause
  \item Mở tệp vdbeamer.tex và biên dich:\\
 - Biên dịch theo LaTeX không có lỗi là cài đặt hoàn chỉnh.\\
 - Nếu có lỗi thì có lẽ thiếu gói lệnh: pgf-1.0.orig.tar và latex-xcolor-2.0.orig.tar, bạn phải tìm và cài lấy.
  \end{itemize}
\end{frame}

\begin{frame}  
\frametitle{2. Cấu trúc lớp Beamer}
  \pause
  \begin{itemize}
  \item Phần đầu phải có\\
$\backslash$documentclass$\{$beamer$\}$\\
$\backslash$usetheme$\{$split$\}$ mẫu màn hình có thể thay split bằng warsaw, bars, boxes, classic, lined, plain, shadow, sidebar, sidebar-tab, tree, tree-bars\\
$\backslash$setbeamercovered$\{$transparent$\}$\\
$\backslash$usepackage[english]$\{$babel$\}$ Bộ chữ cái tiếng Anh.\\
$\backslash$usepackage[vntime]$\{$vntex$\}$ Cài dấu tiếng Việt.
 \pause
 \item Trang đầu tiên của nội dung cần chiếu:\\
 $\backslash$title$\{$Dùng Beamer như thế nào$\}$\\
$\backslash$author$\{$Nguyễn Hữu Điển$\}$\\
$\backslash$institute[ĐHKHTN Hà Nội]$\{$\\
  Khoa Toán - Cơ - Tin học\\
  Đại học Khoa học Tự nhiên Hà nội$\}$
  \end{itemize}
\end{frame}

\begin{frame}  
\frametitle{3. Một trang màn hình}
\pause
\begin{itemize}
\item Trang đầu tiên sau môi trường {\bf document}\\
$\backslash$frame$\{$ $\backslash$maketitle$\}$
\pause
\item Trang khác gồm các lệnh\\
$\backslash$begin$\{$frame$\}$\\  
$\backslash$frametitle$\{$ tiêu đề trang $\}$\\
$\backslash$pause\\
$\backslash$begin$\{$itemize$\}$\\
$\backslash$item ...\\
$\backslash$item ...\\
 $\backslash$end$\{$itemize$\}$\\
 $\backslash$end$\{$frame$\}$  \\
 Chú ý: sau mỗi item hoặc itemize có lệnh  $\backslash$pause
\pause 
 \item Mỗi trang lặp lại cấu trúc này.
  \end{itemize}
\end{frame}
 
\begin{frame}  
\frametitle{4. Soạn thảo và biên dịch}
\pause
\begin{itemize}
\item Các lệnh của \LaTeX{} soạn thảo bình thường.
\pause
\item Các lệnh và môi trường toán cũng không thay đổi gì, ví dụ:
\begin{equation}
f(x)=\begin{cases}
1, \mbox{ nếu }x\ge 0,\\
0, \mbox{ngược lại.}
\end{cases}
\end{equation}
\pause 
 \item Biên dịch từ tệp TeX sang tệp PDF:\\
 - Trong winshell nhấn vào nút PdfTeX  để dịch từ chương trình pdflatex:\\
 Có thể phải đặt lại đường dẫn đến chuopwng trình này như\\
 Option $\rightarrow$ Program call $\rightarrow$ PDFLatex\\
 gõ vào ô exe-File: pdflatex và ô cmd-line: \%s.tex
 \pause 
 \item Chạy chương trình bằng acrobat reader 7.0 là tốt nhất. Bạn cũng có thể cài chương trình này vào trong Winshell ở nút: PDF. Cách làm tương tự phần trên.
  \end{itemize}
\end{frame}

\begin{frame}  
\frametitle{Kết thúc}
\begin{center}
Chúc mọi người may mắn!
\end{center}
\end{frame} 
\end{document}
