\documentclass[12pt,a4paper,openany]{book}
\usepackage{amsmath,amsxtra,amssymb,latexsym, amscd,amsthm}
\usepackage{enumerate}
\usepackage[shortlabels]{enumitem}
\usepackage{indentfirst}
\usepackage[mathscr]{eucal}
\usepackage{color}
\usepackage[utf8]{vietnam}
\usepackage{graphicx}
\usepackage[top=2.5cm, bottom=2cm, left=2cm, right=2cm] {geometry}
\theoremstyle{definition}
\newtheorem*{cau1}{Câu 1 (1,0 điểm)}
\newtheorem*{cau2}{Câu 2 (1,0 điểm)}
\newtheorem*{cau3}{Câu 3 (1,0 điểm)}
\newtheorem*{cau4}{Câu 4 (1,0 điểm)}
\newtheorem*{cau5}{Câu 5 (1,0 điểm)}
\newtheorem*{cau6}{Câu 6 (1,0 điểm)}
\newtheorem*{cau7}{Câu 7 (1,0 điểm)}
\newtheorem*{cau8}{Câu 8 (1,0 điểm)}
\newtheorem*{cau9}{Câu 9 (1,0 điểm)}
\newtheorem*{cau10}{Câu 10 (1,0 điểm)}
\usepackage{pgf,tikz}
\usepackage{mathrsfs}
\usetikzlibrary{arrows}

%\hdashline
%\usepackage{array, arydshln}
%phamtom


\begin{document}
\thispagestyle{empty}
\begin{minipage}{0.3\textwidth}
	\begin{flushleft}
		\begin{center}
			\textbf{BÀI TẬP PP NCKH}\\
			\textit{(Thiết kế đề thi bằng \LaTeX)}\\
			--------------\\
			ĐỀ SỐ 51
		\end{center}
	\end{flushleft}
\end{minipage}
\begin{minipage}{0.7\textwidth}
	\begin{center}
		\textbf{ĐỀ THI THỬ - KÌ THI THPT QUỐC GIA NĂM 2016}\\
		\textbf{Môn thi: TOÁN}\\
		\textit{Thời gian làm bài: 180 phút, không kể thời gian phát đề}\\
		\textbf{-----------}
	\end{center}
\end{minipage}

~\\
\setlist[enumerate,1]{leftmargin=1.4cm}

\begin{cau1} Khảo sát sự biến thiên và vẽ đồ thị $(C)$ của hàm số $y=\dfrac{3-2x}{x-1}$
\end{cau1}

\begin{cau2} Viết phương trình tiếp tuyến của $(C)$ biết tiếp tuyến song song với đường thẳng $\Delta: y=-x+1$\\[-10pt]
\end{cau2}

\begin{cau3}~
	\begin{enumerate}[a)]
		\item Tìm số phức liên hợp của số phức $z$ thỏa mãn $3z+9=2i.\overline{z}+11i$.
		\item Giải hệ phương trình: $\log_{\frac{1}{2}} (x^2+5)+2\log_2(x+5)=0$
	\end{enumerate}
\end{cau3}


\begin{cau4}  Tính tích phân: $I=\int_{0}^{1} x\left(x+e^{z^2}\right)dx$
\end{cau4}
\begin{cau5} Trong không gian $Oxyz$, cho 3 điểm $A(4;-4;3), B(1;3;-1), C(-2;0;1)$. Viết phương trình mặt cầu $(S)$ đi qua các điểm $A, B, C$ và cắt hai mặt phẳng $(\alpha): \mbox{\textit{x}+\textit{y}+\textit{z}+2=0}$ và $(\beta): x-y-z-4=0$ theo hai giao tuyến là hai đường tròn có bán kính bằng nhau.
\end{cau5}
\begin{cau6}~
	\begin{enumerate}[a)]
		\item Giải phương trình: $(\sin x+\cos x)^2=1+\cos x$
		\item Một tổ gồm 9 học sinh nam và 3 học sinh nữ. Cần chia tổ đó thành 3 nhóm, mỗi nhóm 4 học sinh để đi làm 3 công việc trực nhật khác nhau. Tính xác suất để khi chia ngẫu nhiên ta được mỗi nhóm có đúng 1 nữ.
	\end{enumerate}
\end{cau6}
\begin{cau7} Cho khối chóp $S.ABC$ có $SA$ vuông góc với mặt đáy $(ABC)$, tam giác $ABC$ vuông cân tại $B$, $SA=a$, $SB$ hợp với đáy một góc $30^0$. Tính thể tích của khối chóp $S.ABC$ và tính khoảng cách giữa $AB$ và $SC$.
\end{cau7}

\begin{cau8}
	Trong mặt phẳng tọa độ $Oxy$, cho hình chữ nhật $ABCD$ có hình chiếu $B$ lên $AC$ là $E(5;0)$, trung điểm $AE$ và $CD$ lần lượt là $F(0;2), I\left(\dfrac{3}{2};-\dfrac{3}{2}\right)$. Viết phương trình đường thẳng $CD$.
\end{cau8}

\begin{cau9}
	Giải bất phương trình: $\left(2-\dfrac{3}{x}\right)\left(2\sqrt{x-1}-1\right) \geq \dfrac{4-8x+9x^2}{3x+2\sqrt{2x-1}}$
\end{cau9}

\begin{cau10}
	Cho $a, b, c>0$ và thỏa mãn: $c=\min\{a,b,c\}$. Tìm Giá trị nhỏ nhất của biểu thức:
	\[
	P=\sqrt{\dfrac{a}{b+c}}+\sqrt{\dfrac{b}{c+a}}+\dfrac{2\ln\left( \dfrac{6(a+b)+4c}{a+b}\right)}{\sqrt[4]{\dfrac{8c}{a+b}}}
	\]
\end{cau10}
\begin{center}
	\textbf{--------- Hết ---------}
\end{center}


\noindent
Nguồn biên tập:\dotfill\\
Họ và tên học viên: Tăng Lâm Tường Vinh. MSHV: 15 23 015. Lớp: Xác suất - Thống kê\dotfill



\end{document}
 
