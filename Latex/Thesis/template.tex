\documentclass{beamer}
\mode {
\setbeamertemplate{background canvas}[vertical shading][bottom=red!10,top=blue!10]
 
\usetheme{Warsaw}
 
\usefonttheme[onlysmall]{structurebold}
}
 
\usepackage{cmap}
\usepackage{amsmath,amssymb}
\hypersetup{pdfpagemode=FullScreen,bookmarks=false}
\usepackage{colortbl}
\usepackage[utf8]{vietnam}
\setbeamercovered{dynamic}
\usepackage{url}
\DeclareGraphicsRule{*}{mps}{*}{}
\usepackage{graphics}
\usepackage{bbding}
\usepackage{xcolor}
 
\font\timesbhai=utmb8v at 12pt
\font\timesihai=utmbi8v at 12pt
\font\timesmot=utmr8v at 11pt
\font\timesbbon=utmb8v at 14pt
\font\utopiamuoi=putr8v at 10pt
\font\utopiamot=putr8v at 11pt
\font\utopiahai=putr8v at 12pt
\font\utopiabon=putr8v at 14pt
 
\newcommand{\DarkBlue}[1]{\textcolor[rgb]{0,0.08,0.45}{#1}}
 
\newcommand{\trangtieude}[5]{
\title[#1]{\textbf{\textcolor[rgb]{1.00,1.00,0.00}{\timesbbon #1}}\\ }
\subtitle{\textbf{\textcolor[rgb]{1.00,1.00,0.00}{\utopiamuoi #2}}\\ }
\author[#3]{#3}%
%vspace*{-.25cm}
\institute{#4 \and \textcolor[rgb]{0.98,0.00,0.00}{\textbf{\url{#5}}}}
}
 
\newcommand{\slide}[3]{
\begin{frame}
#1 % \transwipe[direction=0]
\frametitle{\utopiamuoi #2}\pause
\utopiamuoi #3
\end{frame}
}
 
\newcommand{\Khoi}[2]{\begin{block}{#1}#2\end{block}}
\newcommand{\Cot}[2]{\column{#1}#2}
 
\begin{document}
\trangtieude{Giải bài thi tuyển sinh Đại học năm 2010}{Môn Toán }{TS Nguyễn Thái Sơn}{Đại học Sư Phạm TP Hồ Chí Minh }{http://hcmup.edu.vn}
\logo{\includegraphics[height=2cm]{flower.png}}
\titlepage
\section*{Nội Dung}
%%%%%%%%%%%%%%%%%%%%% Viet vao day
 
\slide{\transwipe[direction=0]}{I. PHẦN CHUNG CHO TẤT CẢ THÍ SINH (7,0 ĐIỂM)}{
\Khoi{Câu I. (2,0 điểm)}{Cho hàm số $y=x^3-2x^2+(1-m)x+m \quad (1), m$
là số thực.
 
\begin{enumerate}
\item Khảo sát sự biến thiên và vẽ đồ thị của hàm số khi $m=1$.
\item Tìm $m$ để đồ thị của của hàm số $(1)$ cắt trục hoành tại ba điểm
phân biệt có hoành độ $x_1,x_2,x_3$ thoả điều kiện $$x_1^2+x_2^2+x_3^2<4.$$
\end{enumerate}}
}
\slide{\transwipe[direction=0]}{Câu I.}{
 
\Khoi{2.}{Tìm $m$ để đồ thị của của hàm số $(1)$ cắt trục hoành tại ba điểm
phân biệt có hoành độ $x_1,x_2,x_3$ thoả điều kiện: $$x_1^2+x_2^2+x_3^20 \\
g(1) =-m \ne 0
\end{array}
\right. \pause
\Longleftrightarrow \pause
\left\lbrace
\begin{array}{l}
m\ne 0\\
m > -\dfrac14
\end{array}
\right.
$$
}
 
\slide{\transwipe[direction=0]}{Câu I.}{
 
trong đó \pause $x_1=1$, $x_2$ và $x_3$ là nghiệm của phương trình (*).\pause
 
Ta có:\pause $$x_1^2+x_2^2+x_3^2=1+ (x_2+x_3)^2-2x_2x_3= 1+ 1+2m=2+2m$$\pause
 
Do đó:\pause $$x_1^2+x_2^2+x_3^2 < 4 \Longleftrightarrow 2+2m<4 \Longleftrightarrow m<1$$\pause
 
Kết hợp với điều kiện ta có:\pause $$-\dfrac14 <m \> $\cos x \ne 0 $ và $1+\tan x \ne 0 $\\
\> $\Longleftrightarrow$ \>$\cos x \ne 0 \ \text{và}\ \sin x+\cos x \ne 0$.
\end{tabbing}
\pause
 
Ta có: $\sin(x+\frac{\pi}{4})=\dfrac{\sqrt{2}}{2}(\sin x+\cos x)$
và $1+\tan x = \dfrac{\sin x+\cos x}{\cos x}$. \pause
 
Do đó phương trình đã cho trở thành:\pause
 
\begin{align}
1+\sin x+\cos 2x=1 &\Longleftrightarrow -2\sin^2x+\sin x+1=0\nonumber \\
&\Longleftrightarrow \left[ \begin{array}{l}
\sin x=1 \quad \text{(\textit{loại, do điều kiện $\cos x \ne 0$)}} \\
\sin x=-\dfrac{1}{2}
\end{array} \right. \nonumber
\end{align}
}
 
\slide{\transboxin}{Câu II.}{
\begin{columns}
\Cot{.4\textwidth}{
\Khoi{}{Vậy nghiệm của phương trình đã cho là:}
}\pause
\Cot{.5\textwidth}{
\Khoi{}{$$\left[ \begin{array}{l}
x=-\dfrac{\pi}{6}+k2\pi \\ \\
x=\ \ \dfrac{7\pi}{6}+k2\pi
\end{array} \right.
$$}
}
\end{columns}
}
\end{document}